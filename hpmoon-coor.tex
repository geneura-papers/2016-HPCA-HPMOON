%\affiliation{\affilnum{1}Dept. of Computer Architecture and Computer Technology, University of Granada, Spain}

%\corrauth{Pablo Garc\'ia-S\'anchez,
%ETS. Ingenier\'ias en Inform\'atica y Telecomunicaci\'on and CITIC-UGR,
%C/Periodista Daniel Saucedo s/n,
%Universidad de Granada,
%18071, Granada, Spain.}

\documentclass[preprint]{elsarticle}
\usepackage[latin1]{inputenc}
\usepackage[english]{babel}
\usepackage{moreverb,url,color}
\usepackage{algorithm}
\usepackage{algorithmic}
\usepackage{graphicx}
\usepackage{color}
\usepackage{url}
\usepackage[colorlinks,bookmarksopen,bookmarksnumbered,citecolor=red,urlcolor=red]{hyperref}

\begin{document}

\begin{frontmatter}

%%%%%%%%%%%%%%%%%%%%%%%%%%%%%%%   TITLE   %%%%%%%%%%%%%%%%%%%%%%%%%%%%%%%

\title{Parallel multi-objective optimization using selective operator application
  islands}
% New title suggestion
% Should co-evolutionary go somewhere? Are they co-evolutionary? - jj
% PABLO: Renaming again to avoid using the co-evolutionary and
% cooperative
% Ok - JJ

%%%%%%%%%%%%%%%%%%%%%%%%%%%%%%%   AUTHORS   %%%%%%%%%%%%%%%%%%%%%%%%%%%%%%%

\author{P. Garc\'ia-S\'anchez$^1$, J. Ortega$^2$, J. Gonz\'alez$^2$, P. A. Castillo$^2$ and J.J. Merelo$^2$}
\ead{pablo.garciasanchez@uca.es, \{jortega, jesusgonzalez, pacv, jmerelo\}@ugr.es}
\address{
$^1$ Department of Computer Science and Engineering. ESI. University of C\'adiz, Spain\\
$^2$ Department of Computer Architecture and Computer Technology.\\ ETSIIT - CITIC. University of Granada, Spain\\
}


\begin{abstract}
  % First, state the problem!!!! - JJ Pablo: ok!
Real-world optimization problems usually require to deal with several objectives.
Multi-objective optimization methods have gained interest in the last
years because of the performance they can achieve solving this kind of problems. 
Many of them usually require a high-dimensional
decision space, and multi-objective Evolutionary Algorithms have been
successfully used in the past. Due to the large computational cost
these algorithms require, different parallelization and distribution
methods have been proposed, with different parallel and distributed
computer architectures to cope with them. 

One of the most extended
methods are the distributed co-evolutionary algorithms, where each
processor deals only with a part of the decision space, that is, the
individuals belonging to the same sub-population (island) explore the
same subset of decision variables. However, these methods only works with decomposable problems.

A similar idea is presented here to deal with un-decomposable problems, the selective operator application:
instead of sharing parts of individuals, each processor applies the variation 
operators to a specific subset of the whole individual. We have previously proved 
that extending the length of these subsets, and therefore, sharing some search 
space between islands, implies an improvement in the results.


In this paper, a new method to
automatically adapt the size of the overlapping, or shared sections,
of the individuals has been compared to other collaborative
evolutionary techniques. Different number of islands, chromosome sizes
and problems have been used. The analysis of the obtained experimental
results, by using different metrics, shows that selective operator application
approaches can provide statistically significant improvements with
respect to the base algorithm, being the method to automatically adapt
the overlapping size the one to obtain the best performance during the
same execution time. Results also show that the relation of the number
of islands (subpopulations) to the length of the chromosome (number of
decision variables) is also a relevant factor to determine the most
efficient alternative to distribute the decision variables. 
\end{abstract}


\begin{keyword}
Multi-objective algorithms \sep NSGA-II  \sep distributed evolutionary algorithms \sep Island model
\end{keyword}

\end{frontmatter}





\section{Introduction}

%Introduction should start in the same way as the abstract. And
%starting with something that is not really applicable here, talking
%about general parallel evolutionary algorithms, is even worse.
% Right way (IMNSHO)
% 1. Multiobjective problems are
% 2. They usually require a good amount of computation → parallel
% distributed
% 3. One way to solve them is to use MOEAs _because_ they can be
% easily parallelized
% 4. But the easy paralellization is not necessarily efficient.
% 5. We have proposed the HPMOON (or whatever) technique to improve
% it. In paper (whatever) we proved that (whatever)
% 6. In this paper we will be proving whatever - JJ FERGU: Addressing JJ order

%1. MOP ARE
Multi-objective optimization problems (MOP) are those where several
objectives have to be simultaneously optimized
\citep{Mora13paretobased}. Solving a MOP implies maximizing or minimizing
a function composed of several cost functions (one per objective). In
these problems the aim is to obtain a set of solutions that are better
than the rest, considering all the objectives; this set is known as
the Pareto Front (PF). The solutions in this set are {\em non-dominated},
that is, none is better than the others for all the objectives.


%2 COMPUTATION -> PARALLEL
These kind of problems require a high demand of computational time, leading to the apparition of easily parallelizable methods to solve them \citep{Luna15Survey,Mukhopadhyay14Survey,Chavez15MO,Hidalgo16residualstress}.
%3 ONE WAY IS TO USE MOEAS BECAUSE
One example of easily parallelizable algorithms to solve MOPs are the Multi-Objective Evolutionary Algorithms (MOEAs), a sub-set of Evolutionary algorithms (EAs). EAs \citep{DBLP:series/ncs/EibenS15} are bio-inspired meta-heuristics that can be effectively used to find nearly optimal solutions for optimization problems. Usually an EA starts by generating a set of random solutions, called \emph{population}, following a user-defined description. Then, it evaluates each candidate solution, called \emph{individual}, assigning it a \emph{fitness} value, that describes how good the individual is, with regard to the target problem. New solutions are then generated by the application of \emph{operators} that either mutate a single solution or recombine different existing solutions. After each iteration, called \emph{generation}, the least fit individuals are removed, and the process continues until a user-defined stop condition is met. One of the most well-known and referenced MOEAs \citep{Dorronsoro13superlinear} is the {\em
  Non-dominated Sorting Genetic Algorithm} (NSGA-II)
\citep{Deb00NSGAII}. Classical genetic operators are applied to all
the individuals, and divided into different ranks (based on the
dominance) to be selected for the next generations. 

%4 BUT THE EASY PARALLELIZATION IS NOT NECESSARILY EFFICIENT

Different approaches have been used to parallelize EAs, as each individual can be considered as an
independent unit \citep{Alba13parallel}. Classic methods, such as the
global parallel EAs (Master-Slave), or the spatially structured
algorithms (Island model or Cellular EAs) have been applied
successfully in the past \citep{Folino03cellular,Alba02Parallelism}.
However, in the case of MOEAs, these
approaches \citep{Luna15Survey} need to deal with the whole solution set, the PF.
 This imply to use different distribution
and sharing mechanisms, as there exist a tension between the speedup
achievable from parallelization and the need to globally recombine the
results to accurately identify the PF
\citep{Branke04Parallelizingcone}. 

MOPs usually require a high number of variables, meaning that the MOEAs need to deal with large individuals and spend significant extra time for crossover, mutation and migration. Different authors have proposed methods to divide the decision space (the chromosome) to improve the performance and solution quality. In this aspect, the co-evolution model is a dimension-distributed model where a high-dimensional problem is divided into lower dimensional problems \citep{Gong15models,Tonda12cooperative}, and evolved separately. One example of application of this technique was described in \citep{Kimovski15Parallel}. The method presented in that works involves different workers that evolve sub-populations created and recombined by a master process, which performs different recombination alternatives from the parts returned by worker processes. A high dimensional problem was used to compare these
alternatives. In \citep{Dorronsoro13superlinear}, a distributed
co-evolutionary island model was used.  Although in both papers
significant speedups were attained, only a low number of nodes were
employed (8). However, when the number of islands is increased, the division
of each section of the chromosome becomes smaller, and the scalability
may be affected by obtaining lower quality solutions in the same
amount of time. 





%5 WE HAVE PROPOSED HPMOON



A similar approach used to solve this kind of problems is the {\em selective operator application} (SLA). In this case, each island deals with the whole chromosome, but only modifies a fragment of it in the crossover and mutation phase depending on the number of islands, using the whole chromosome for fitness calculation. This allows to deal with un-decomposable problems.
In our preliminary work \citep{Garcia16hpmoon} we used this method in a preliminary experimental setup, although it is worth mentioning we are establishing its name and acronym in this paper for the first time.  We proved
that applying the variation operators only over specific sections of the whole chromosome improves the quality of the solutions in the same computing time for a multi-objective island-based algorithm.
Moreover, instead of making every island focus on a disjoint subset of the
chromosome, the usage of overlapped (shared) sections of the
chromosome 
can improve the quality of the solutions 
when the number of islands is increased. We discovered that the most
adequate overlapping size may vary depending on a relation between the
number of islands and the individual size. However, the experiments
were conducted in a limited experimental system: mono-processor and synchronous island model. This
motivates us to continue this research line using a more complete environment: a real cluster with up to 128 nodes, and a more complete parameter setting. Besides comparing previous methods in this new experimental setup, in this paper we propose
a new method that automatically adapts the size of the overlapped
sections depending on the number of available islands, comparing it with previous versions. 
% We have to
                                % emphasize what is actually done. Is
                                % this an addition to the old paper,
                                % or a complete rewrite? - JJ FERGU: rewritten previous paragraph


%6 IN THIS PAPER WE WILL BE PROVING

The hypothesis of this paper is that using overlapped sections of the
chromosome for the SLA, and obtaining the number of sections automatically from the number of available islands, in a
multi-objective island-based algorithm, in a real computational environment, can improve the
quality of the solutions in the same computing time with respect to
the baseline, disjoint and partly overlapped methods. %That was the
                                %hypothesis of the previous
                                %paper. This paper has a new
                                %hypothesis, right? - JJ FERGU: no, this is the hypothesis of this paper (comparing adaptive with partly overlapped), but clarifying it
In order to prove this claim, our new adaptive method has been
compared against different overlapping islands schemes: a
baseline version of a distributed NSGA-II \citep{Deb00NSGAII}, a
SLA version with disjoint sections and a SLA
fixed-size overlapped version. %What was done in the previous paper?
                               %Are we re-using the results in that
                               %paper? - JJ FERGU: explained before that we are showing new results.
Different benchmark problems and different large number of islands
have been used. Results show that this new technique can improve
different quality indicators in the same amount of time when the
problem size is large.  % Shouldn't this start with "We are trying to
                        % prove..." - JJ FERGU: Already writed in previous paragraph

The rest of the paper is organized as follows: after the State of the Art in distributed and co-evolutionary MOEAs, 
the used methodology and the compared algorithms are described in Section \ref{sec:co-evo}. 
Then, the results of the experiments are presented (Section \ref{sec:res}), Finally, the conclusions and future work are discussed.


%%%%%%%%%%%%%%%%%%%%%%%%%%%%%%  STATE OF THE ART  %%%%%%%%%%%%%%%%%%%%%%%%%%%%%
%
\section{State of the Art}
\label{sec:soa}




% Parallel multiobjective optimization is the theme. And there are
% papers that do that. Please include them in the state of the art:
% dennis2003parallel (added to bibliography). FERGU: why this paper in particular? It parallelizes the fitness function steps and it is not multi-objective. It does not even describes the GA in detail.

Since the early 2000s distributed and parallel Evolutionary
Algorithms have been used mostly to leverage systems such
as clusters or grids \citep{Talbi08Parallel}, since a first-order
approach to EA parallelization is as easy as dividing the population
and add a simple communication mechanism among them. But in the case of
MOEAs, the distribution and parallelization is not as direct as in
single-objective EAs. This is because in different steps of the
algorithm a whole set of dependent solutions, the Pareto Front, should
be managed as a whole, spending time in gathering all individuals from
the different processors or islands. 

To solve this issue, some authors have proposed using Master-Slave approaches. For example,  \citep{Durillo08masterslave} compared different master-slave approaches: synchronous generational, asynchronous generational and asynchronous steady-state, being the latter the most promising option. The method proposed by Hiroyasu \citep{Hiroyasu07discussion} generates offspring depending on the computational power.

%This is an improvement over the previous? What is the relationship? FERGU: linked
Other kind of approaches have also been explored. The work by Deb et al. \citep{Deb03distributed} was one of the first approaches for distributed MOEAs (dMOEAs). In that work, the dominance of the solutions was divided into the islands using a coordinate transformation. Their authors concluded that dividing the search space is a good idea, although achieving this is not trivial. The division of the search space has been explored by other researchers, for example, dividing the population in elite and search sub-populations \citep{Wang09parallel}, or separation in processors by objective \citep{Xiao03specialized}. Other authors, such as Martens \citep{Martens13asynchronous} use migration to accept individuals based on diversity, and migrating from not-crowded areas.

% Is this really cooperative co-evolution? - JJ FERGU: You mean the Dorronsoro one? The title says it
Addressing high-dimensional problems with cooperative co-evolution has
also been studied in several works with approaches closer to the one
presented here. The approach to focus on a portion of the chromosome,
as in our overlapped method, was firstly used in the work by
Dorronsoro et al. \citep{Dorronsoro13superlinear}, obtaining
super-linear performance 
in several instances. This approach has also been used by Kimovski et al. \citep{Kimovski15Parallel}, but using a
master-slave method that splits the population into several processors. As in previous work, each node
runs a parallel MOEA that only affects some portion of 
the individuals and the master
process receives all the sub-populations to be combined every certain number of generations. Up to 8 processors
were used in the experimental setup, and several
combination alternatives were compared. The main difference of our work with respect to previous works is that our approach does not 
broadcast all solutions to all islands for recombination, but only one solution to a random island, needing less
communication time. Moreover, Dorronsoro or Kimovsky approaches limited the maximum number of islands to 8, while in this paper we have
used up to 128 islands. 

Giving the responsibility of different parts of the search space to
different modules, has also been explored lately by some
researchers. For example, some authors focus on the solution space
modeling the PF by dividing it in different clusters, allowing an
easily parallelizable {\em divide and conquer} approach
\citep{cheng2015adaptive}. Even other types of metaheuristics, such as
the Multi-Objective Ant Colony Optimization (MOACO), can rely on
island models to divide the search space depending on the node to
achieve better results \cite{Mora13paretobased}. 

In our previous work \citep{Garcia16hpmoon} we used some of
the aforementioned ideas to compare two different dMOEAs. The first
one divided the chromosome in $P$ sections, being $P$ the number of
islands. Every island $p$ only performed the mutation and crossover in
that part ($p_{th}$) of the chromosome (the selective operator application), while the fitness was
calculated using the whole individual. After a certain number of
generations, individuals were migrated randomly to other islands. The
performance metrics were calculated at the end of the run. The second
method, selective operator application with overlapping islands, used the $p_{th-1}$ and
$p_{th+1}$ sections of each chromosome, besides the $p_{th}$
part. During the same amount of time both methods obtained better
results than a baseline algorithm that dealt with the whole chromosome
in each island for crossover and mutation. We discovered that the
performance using one or another method depends on the number of
sections of the individuals and the number of islands used. This
motivated us to find a new adaptive method to select this number of
sections of the chromosome to use, depending on the number of
islands. Besides, previous experiments were performed on a single
processor synchronous island model with a limited number of islands
(8, 32 and 128). In this paper,  8, 16, 32, 64 and 128 islands have
been used, and this time, the experiments have been run on a real
cluster. Therefore, at the same time we are proposing an adaptive
method to divide the individual search space, and also validating the
previous methods. 


We will describe how we have tested our approach in the next Section.




%
%%%%%%%%%%%%%%%%%%%%%%%%%%%%%%  METHODOLOGY  %%%%%%%%%%%%%%%%%%%%%%%%%%%%%
%

\section{Methodology} % This should be called
                                % methodology. The object of the
                                % section is to explain the
                                % methodology used to test the
                                % problem, in order to do that we have
                                % tested different versions that prove
                                % that our methodology works for the
                                % precise reasons we think it works- JJ FERGU: Done, and explained
\label{sec:co-evo}

The objective of this section is to explain the methodology we have followed to
compare the different versions of the overlapping schemes.

In this paper we analyse several SLA algorithms we have
implemented by using a parallel multi-objective baseline algorithm
as a basis for comparisons. Our proposed methods, using different overlapping schemes, %methods or method?
                                %Which methods? - JJ Fergu: clarifying
are based on NSGA-II, as almost all the previous works discussed
before
\cite{Dorronsoro13superlinear,Durillo08masterslave,Hiroyasu07discussion,Deb03distributed,Xiao03specialized,Wang09parallel,Martens13asynchronous}. % This is SoA stuff - JJ FERGU: Yes, I added it here againg because a reviewer wasn't very attentive
Therefore, we have used a basic distributed NSGA-II algorithm without
overlapping sections as the baseline (B).

%This probably should deserve a "Algorithm" environment too - JJ FERGU: You mean the LaTeX one? I've used it in the pseudocode figures
This basic distributed algorithms spread the population among $P$
islands; after a fixed number of generations,
one individual in a given island is migrated to another random island,
thus avoiding to synchronize the global Pareto Front (PF) every
certain number of generations as other methods described in Section 2
do. At the end of the run, the Pareto Fronts of all islands are
aggregated into a new one, and the quality measures are evaluated. 
Different  SLA alternatives can be devised to evolve the
subpopulations according to the decision space to be explored by each
islands. Here we will consider three methods in which each island only
performs crossover and mutation in specific sections of the
individuals. 



%This algorithm is a regular NSGA-II algorithm distributed along a number $P$ of islands. After a fixed number of generations, one individual is migrated to another random island. At the end of the run, PFs of all islands are aggregated in a new one in order to compute the quality measures. This baseline method has been chosen as it does not require synchronization of the global PF every certain number of generations, as other methods described in Section 2 do.
%% some justification on why this was chosen - JJ PABLO: Added 
%
%% And some liaison to next subsection - JJ
%This method will be compared to different co-evolutionary approaches that are similar to the baseline, with the exception on the decision space to explore in each island.
%Three different co-evolutionary methods have been used. In all these methods, each island only performs crossover and mutation in specific sections of the individuals.

As in the baseline (B), an individual is migrated to another random
island after a fixed number of generations. In the new island, this
individual will be considered as one of the others in the island,
crossed and mutated in the same way, depending on the island
identifier. Note that, differently from other works such as the ones
described by Talbi et al. \citep{Talbi08Parallel} all the islands deal
with complete chromosomes for fitness calculation, so our approach can
deal with decomposable and no decomposable problems. %Don't understand what you say
                                %here. What you are saying is that our
                                %approach can't deal with decomposable
                                %problems, actually. - JJ FERGU: you are right, fixed

The methods we have tested are presented in the next subsections. 

\subsection{SLA with disjoint islands (D)} 
In this approach, each individual of size $L$ is split into $P$ chunks
of size $L/P$. Every island $p$ only performs crossover and mutation
on the $p_{th}$ part of the individuals. Figure \ref{fig:disjoint}
describes this approach.   
%
\begin{figure*}[h!tb]
\centering
\includegraphics[width=12cm]{islandDisjoint.jpg}
\caption{SLA with disjoint islands (D): every island $p$ only modifies the $p_{th}$ components (in grey) of the individuals.}
\label{fig:disjoint}
\end{figure*}

\subsection{SLA with overlapping islands (O)}
This approach is similar to the previous one, but every island also uses the $p+1$ and $p-1$ (module size) chunks of the individual for crossover and migration. Therefore, some kind of overlapping of the crossed and mutated parts exists between islands. Figure \ref{fig:overlapping} shows the affected parts of the individuals in
each island. 

\begin{figure*}[h!tb]
\centering
\includegraphics[width=12cm]{islandNoDisjoint.jpg}
\caption{SLA with overlapping islands (O): every island $p$ modifies the  $p+1$,
  $p_{th}$ and $p-1$  components (in grey) of the individuals.}
  \label{fig:overlapping}
\end{figure*}

\subsection{Adaptive SLA with overlapping islands (A)} 
As in the previous method, the sections to address are overlapped, but
instead using one extra section on each side of the $p$ section ($p+1$
and $p-1$), it uses $c$ fragments on each side ($p+c$ and $p-c$),
being $c$ a value depending on the number of islands.
% That's not adaptive. It's simply variable - JJ FERGU: it depends on the number of islands. What term do you suggest?
As a first
approach to automatically calculate this value, the results shown in
\citep{Garcia16hpmoon} have been used as a base to obtain $c$. In that
work, the overlapped method ($c=1$) obtained better results when the
number of islands were $>$8. On the contrary, when the number of
islands is small, no overlapping is necessary. So, we have used this
knowledge to create the formula $c=round(0.2*P-1)/2$ to calculate the
extra sections to overlap. Therefore, when $P=8$ then $c=0$
(equivalent to D), when $P=16$ then $c=1$ (equivalent to O), and so
on. As mentioned, this is only a first approach to obtain $c$, and future work will study new methods to calculate it. Figure \ref{fig:adaptive} explains this method. 

\begin{figure*}[h!tb]
\centering
\includegraphics[width=12cm]{islandAdaptive.jpg}
\caption{Adaptive SLA with overlapping islands (A): every island $p$ modifies the  $p+c$,
  $p_{th}$ and $p-c$  components (in grey) of the individuals. $c$ is calculated depending on the number of islands. In this case, $c=2$.}
  \label{fig:adaptive}
\end{figure*}
% --------------------------------------------------------------

Pseudo-code of the proposed algorithms is described in Algorithm \ref{alg:EA}.

\begin{algorithm}[htb]

\begin{algorithmic}
\STATE \textit{//In each island $i$}
\STATE population $\gets$ initialisePopulation()
\STATE \textit{//Select which part modify in this island $i$}
\STATE section $\gets$ getComponentsToModify($i$)
\IF {type = Baseline}
	\STATE $c=P/2$
\ELSIF {type = Disjoint}
	\STATE $c=0$
\ELSIF {type = Overlapping}
	\STATE $c=1$
\ELSIF {type = Adaptive}
	\STATE $c=round(0.2*P-1)/2$
\ENDIF

\WHILE {stopping criterion not met}
    \STATE parents $\gets$ nsgaSelection(population)
    \FORALL{each 2 chromosome in parents}
    	\STATE \textit{//Crossovers only the part for island $i$ (c components left and right)}
    	\STATE children  $\gets$ crossover(parentA[i-c:i+c],parentB[i-c:i+c])
    	\STATE offspring $\gets$ offspring + children
    \ENDFOR
    \FORALL{chromosome in offspring}
    	\IF {mutationProb $>$ random()}
    		\STATE \textit{//Mutates only the part for island $i$}
    		\STATE chromosome $\gets$ mutation(chromosome[i-c:i+c])
    		\STATE offspring $\gets$ chromosome + offspring
    	\ENDIF
    \ENDFOR
    \STATE population $\gets$ population + offspring
    

    \IF {time to migrate}
      \STATE migrants $\gets$ binaryTournament(population)
      \STATE remoteBuffer $\gets$ getRandomIslandBuffer()
      \STATE remoteBuffer.send(migrants)
    \ENDIF

    \IF {localBuffer.size $\neq$ zero}
      \STATE immigrants $\gets$ localBuffer.read()
      \STATE population $\gets$ population + immigrants
    \ENDIF
\ENDWHILE

\STATE sendIslandParetoFrontToServer()
\STATE \textit{Final Pareto Front from all islands is returned by the server}
\end{algorithmic}

\caption{Pseudo-code of the used EA in every island: a distributed NSGA-II algorithm }
\label{alg:EA}
\end{algorithm}



%%%%%%%%%%%%%%%%%%%%%%%%%%  EXPERIMENTS AND RESULTS  %%%%%%%%%%%%%%%%%%%%%%%%%%
%
\section{Experiments and Results}
\label{sec:res}

This section describes the quality indicators used and the experimental setup. The chosen quality indicators, are:

\begin{itemize}
\item Hypervolume (HV): measures the area formed by all non-dominated solutions found with respect to a reference point. Higher values imply better quality of the PF. Figure \ref{fig:hypervolume} clarifies how the hypervolume is calculated.
\item Inverted Generational distance (IGD): calculates the distance of the obtained set of solutions to the optimal PF. Therefore, this metric requires the optimal PF found in the literature, or the theoretical one. In this metric, the lower the better. %Pablo: theoretical?
\item Spread (S): Measures the spread between solutions, taking into account the euclidean distance between consecutive solutions. As in previous metric, the lower value, the better, as it implies solutions distributed along all the PF.
\end{itemize}

\begin{figure}
\centering
\includegraphics[width=6cm]{hypervolume.jpg}
\caption{Hypervolume calculation. Areas from Pareto Front solutions (black points) to a reference point (white point) are added.}
\label{fig:hypervolume}
\end{figure}


We have chosen those metrics because they have been used extensively, especially in some of the papers presented in Section 2 \cite{Dorronsoro13superlinear,Durillo08masterslave,Hiroyasu07discussion,Wang09parallel,Martens13asynchronous}.

%\subsection{Benchmark problems}


The previously described four approaches have been run with two
different chromosome lengths ($L$): 512 and 2048. Different number of
islands ($P$) have also been compared: 8, 16, 32, 64 and 128. This
maximum number of islands have also been  used in previous work in the
literature \citep{Martens13asynchronous}. The crossover and mutation
chosen, SBX and polynomial, have also been  used previously by other
authors in \citep{Durillo08masterslave}.  % SBX is first mentioned
                                % here. Has to explain - JJ

ZDT \citep{zdt2000a} has been chosen as a benchmark, since it is the most widely used in this area \citep{Deb03distributed,Martens13asynchronous,Wang09parallel,Durillo08masterslave}. The optimal PF distribution used for comparison has 1000 solutions \footnote{Optimal PFs are available at:   \url{http://www.tik.ee.ethz.ch/sop/download/supplementary/testproblems/}.}. 





The criterion used for terminating an experiment has been the running
time: 25 seconds for dimension 512 and 100s (four times more) for
2048. We have used time instead of the number of evaluations firstly
because our hypothesis argues that the time saved in crossover and
mutation can be spent on improving the sub-populations and more
operations and migrations can be achieved. Also, we are using
different number of islands (with different sub-population sizes) and that
could lead to different execution times, so it would be difficult to
compare different times and quality of solutions at the same time. 

\begin{table*}
\begin{center}
\begin{tabular}{|c|c|}
\hline
{\em Parameter Name} & {\em Value} \\ \hline
Global population size ($N$) & 1024 \\ \hline
Selection type & Binary Tournament Selection \\ \hline
Replacement type & Generational \\ \hline 
Crossover type & SBX \\ \hline
Mutation  type & Polynomial\\ \hline
Mutation probability & 1/$L$ \\ \hline
Individuals per migration & 1 \\ \hline
Generations between migration & 5 \\ \hline
Selection for migration & Binary Tournament\\ \hline
Runs per configuration & 30 \\ \hline \hline
Number of islands ($P$) & 8, 16, 32, 64 and 128 \\ \hline
Chromosome size ($L$) & 512 and 2048 \\ \hline
Execution time (s) & 25 (for 512) and 100 (for 2048) \\ \hline \hline
\end{tabular}
\caption{Parameters and operators used in the experiments.}
\label{tab:parameters}
\end{center}
\end{table*}

The ECJ framework \citep{ECJ} has been used to run the
experiments. Specific operators have been developed as new modules for
ECJ, and they can be downloaded from our GitHub repository under a
LGPL V3 License
\footnote{\url{https://github.com/hpmoon/hpmoon-islands}}. 
%\footnote{\url{https://ANONYMOUS-REPOSITORY}}. 
The
island model has been executed asynchronously, using the ECJ distributed inter-population exchange
 model, in a cluster of 16 nodes, each one with 16 Intel(R) Xeon(R) CPU E5520
@2.27GHz processors, 16 GB RAM, Broadcom NetXtreme II BCM5716 1000Base-T (C0) PCI Express network cards, CentOS 6.8 and Java Version 1.8.0\_80. Islands have been first distributed among the nodes. That means that, for example, the run of 8 islands requires 8 different nodes, instead of the usage of the 8 processors of a single node.


Different metrics, explained in the previous section, have been used
to calculate the quality of the obtained PFs in each configuration. As
some of the metrics  (such as HV) require a reference point to be
calculated, the point (1,9) has been chosen as reference, as none of
the generated PFs in all runs are dominated by it. Metrics are then
normalized with respect to that point. A Kruskal-Wallis significance
test has been performed to the metrics of all runs of the
configurations, as the Kolmogorov-Smirnov test detected non-normal
distributions. The average results for each configuration are shown in
Table \ref{tab:results512} (for 512 dimensions) and
\ref{tab:results2048} (for 2048). As previously explained, when $P=8$
the results of A are equivalent to the obtained with D (because
$c=0$), and when $P=16$ they are equivalent to O ($c=1$). 


\begin{table*}
\centering
\resizebox{14cm}{!}{
\begin{tabular}{|c||c|c|c|c||c|c|c|c||c|c|c|c||}
\hline
	&	\multicolumn{4}{|c|}{HV}													&	\multicolumn{4}{|c|}{Spread}														&	\multicolumn{4}{|c|}{IGD}														\\ \hline
%\multicolumn{13}{|c|}{2048 dimenOionO}																																													\\ \hline
\#Islands	&	B		&	D		&	O			&	A			&	B		&	D			&	O			&	A			&	B		&	D		&	O		&	A					\\ \hline
\multicolumn{13}{|c|}{ZDT1}																																													\\ \hline
8	&	0.900		&	0.816		& \textbf{	0.926	}		&	Equiv. D			& \textbf{	0.729	}	&	1.041			&	0.898			&	Equiv. D			&	0.013		&	0.029		& \textbf{	0.007	}	&	Equiv. D					\\
16	& \textbf{	0.880	}	&	0.675		&	0.839			&	Equiv. O			& \textbf{	0.721	}	&	0.897			&	0.983		D	&	Equiv. O			& \textbf{	0.017	}	&	0.062		&	0.025		&	Equiv. O					\\
32	& \textbf{	0.829	}	&	0.618		&	0.698			&	0.790			& \textbf{	0.763	}	&	0.879			&	0.886		D	&	0.913	DO		& \textbf{	0.027	}	&	0.076		&	0.056		&	0.036					\\
64	& \textbf{	0.769	}	&	0.596		&	0.628			&	0.736			& \textbf{	0.820	}	&	0.897			&	0.872			&	0.890	DO		& \textbf{	0.040	}	&	0.080		&	0.071		&	0.047					\\
128	& \textbf{	0.707	}	&	0.585		&	0.598			&	0.668			& \textbf{	0.850	}	&	0.956			&	0.883			&	0.910	O		& \textbf{	0.053	}	&	0.083		&	0.079		&	0.062					\\ \hline
\multicolumn{13}{|c|}{ZDT2}																																													\\ \hline
8	&	0.830		&	0.786		& \textbf{	0.851	}		&	Equiv. D			& \textbf{	0.866	}	&	0.996			&	0.987		D	&	Equiv. D			&	0.023		&	0.037		& \textbf{	0.017	}	&	Equiv. D					\\
16	& \textbf{	0.810	}	&	0.601		&	0.776			&	Equiv. O			& \textbf{	0.916	}	&	0.990			&	1.005		D	&	Equiv. O			& \textbf{	0.029	}	&	0.091		&	0.040		&	Equiv. O					\\
32	& \textbf{	0.756	}	&	0.498		&	0.617			&	0.682			& \textbf{	0.976	}	& \textbf{	0.976	}	B	& \textbf{	0.978	}	BD	&	1.054			& \textbf{	0.045	}	&	0.119		&	0.086		&	0.068					\\
64	& \textbf{	0.668	}	&	0.451		&	0.504			&	0.601			&	1.002		& \textbf{	0.976	}		& \textbf{	0.974	}	D	&	1.025	B		& \textbf{	0.070	}	&	0.132		&	0.117		&	0.091					\\
128	& \textbf{	0.576	}	&	0.434		&	0.452			&	0.526			&	1.002		&	1.023			& \textbf{	0.978	}		&	1.015	BD		& \textbf{	0.096	}	&	0.137		&	0.132		&	0.110					\\ \hline
\multicolumn{13}{|c|}{ZDT3}																																													\\ \hline
8	&	0.928		&	0.838		& \textbf{	0.951	}		&	Equiv. D			& \textbf{	0.869	}	&	1.028			&	0.983		D	&	Equiv. D			& \textbf{	0.008	}	&	0.018		&	0.005		&	Equiv. D					\\
16	& \textbf{	0.903	}	&	0.715		&	0.870			&	Equiv. O			& \textbf{	0.846	}	&	0.899			&	0.960			&	Equiv. O			& \textbf{	0.011	}	&	0.032		&	0.014		&	Equiv. O					\\
32	& \textbf{	0.857	}	&	0.655		&	0.737			&	0.824			& \textbf{	0.845	}	&	0.879			&	0.873		D	&	0.881	DO		& \textbf{	0.016	}	&	0.039		&	0.030		&	0.019					\\
64	& \textbf{	0.796	}	&	0.632		&	0.662			&	0.761			& \textbf{	0.859	}	&	0.887			&	0.885		D	&	0.879	BDO		& \textbf{	0.023	}	&	0.042		&	0.038		&	0.027					\\
128	& \textbf{	0.738	}	&	0.620		&	0.633			&	0.705			& \textbf{	0.882	}	&	0.968			& \textbf{	0.888	}	B	&	0.911			& \textbf{	0.029	}	&	0.044		&	0.042		&	0.033					\\ \hline
\multicolumn{13}{|c|}{ZDT6}																																													\\ \hline
8	&	0.268		&	0.223		& \textbf{	0.298	}		&	Equiv. D			& \textbf{	0.988	}	& \textbf{	0.972	}	B	& \textbf{	0.996	}	B	&	Equiv. D			& \textbf{	0.173	}	&	0.191		&	0.160		&	Equiv. D					\\
16	& \textbf{	0.243	}	&	0.113		&	0.219			&	Equiv. O			& \textbf{	0.995	}	& \textbf{	0.987	}	B	& \textbf{	0.987	}	BD	&	Equiv. O			& \textbf{	0.184	}	&	0.240		&	0.193		&	Equiv. O					\\
32	& \textbf{	0.196	}	&	0.071		&	0.115			&	0.162			& \textbf{	0.998	}	& \textbf{	0.989	}	B	& \textbf{	0.988	}	BD	&	1.005	B		& \textbf{	0.204	}	&	0.257		&	0.238		&	0.220					\\
64	& \textbf{	0.145	}	&	0.056		&	0.075			&	0.120			&	0.996		&	0.986		B	& \textbf{	0.983	}	D	&	0.998	B		& \textbf{	0.226	}	&	0.264		&	0.255		&	0.235					\\
128	& \textbf{	0.104	}	&	0.049		&	0.055			&	0.084			&	0.993		&	1.007			& \textbf{	0.979	}		&	0.998	BD		& \textbf{	0.244	}	&	0.266		&	0.263		&	0.251					\\ \hline

\end{tabular}
}
\caption{Average quality metrics obtained after 30 runs per configuration, for the 4 methods compared: baseline (B), disjoint (D), overlapped (O) and adaptive overlapped (A), using a chromosome length of 512 dimensions. Acronyms next to values indicate that there is not significant difference with respect to that method for that value. Best values are marked in bold.}
\label{tab:results512}
\end{table*}



\begin{table*}
\centering
\resizebox{13cm}{!}{
\begin{tabular}{|c||c|c|c|c||c|c|c|c||c|c|c|c||}
\hline
	&	\multicolumn{4}{|c|}{HV}													&	\multicolumn{4}{|c|}{Spread}														&	\multicolumn{4}{|c|}{IGD}														\\ \hline

\#Island	&	B		&	D		&	O			&	A			&	B		&	D			&	O			&	A			&	B		&	D		&	O		&	A					\\ \hline
\multicolumn{13}{|c|}{ZDT1}																																													\\ \hline
8	&	0.891		& \textbf{	0.953	}	&	0.937			&	Equiv. D			& \textbf{	0.681	}	& \textbf{	0.635	}	B	&	0.661		D	&	Equiv. D			&	0.015		& \textbf{	0.002	}	&	0.005		&	Equiv. D					\\
16	&	0.884		&	0.850		& \textbf{	0.942	}		&	Equiv. O			& \textbf{	0.705	}	&	0.908			& \textbf{	0.670	}	B	&	Equiv. O			&	0.016		&	0.022		& \textbf{	0.004	}	&	Equiv. O					\\
32	&	0.851		&	0.674		&	0.859		B	& \textbf{	0.900	}		& \textbf{	0.754	}	&	0.868			&	0.826		D	& \textbf{	0.763	}	B	&	0.023		&	0.062		&	0.020	B	& \textbf{	0.012	}				\\
64	& \textbf{	0.800	}	&	0.608		&	0.697			& \textbf{	0.824	}	B	& \textbf{	0.808	}	&	0.880			& \textbf{	0.861	}	B	& \textbf{	0.823	}	B	&	0.033		&	0.078		&	0.056		& \textbf{	0.027	}				\\
128	&	0.735		&	0.582		&	0.613			& \textbf{	0.745	}		& \textbf{	0.841	}	&	0.888			&	0.878		D	&	0.865		O	&	0.047		&	0.084		&	0.075		& \textbf{	0.043	}				\\ \hline
\multicolumn{13}{|c|}{ZDT2}																																													\\ \hline
8	&	0.832		& \textbf{	0.895	}	&	0.869			&	Equiv. D			& \textbf{	0.849	}	& \textbf{	0.886	}	B	&	0.853		D	&	Equiv. D			&	0.023		& \textbf{	0.006	}	&	0.013		&	Equiv. D					\\
16	&	0.831		&	0.833	B	& \textbf{	0.884	}		&	Equiv. O			& \textbf{	0.810	}	&	1.001			& \textbf{	0.802	}	B	&	Equiv. O			&	0.023		&	0.022	B	& \textbf{	0.009	}	&	Equiv. O					\\
32	&	0.800		&	0.628		&	0.800		B	& \textbf{	0.817	}		& \textbf{	0.848	}	&	0.974			&	0.983		D	&	0.908			&	0.031		&	0.082		&	0.032	B	& \textbf{	0.027	}				\\
64	& \textbf{	0.729	}	&	0.491		&	0.623			&	0.716			& \textbf{	0.909	}	&	0.967			&	0.979		D	& \textbf{	0.997	}	BD	& \textbf{	0.052	}	&	0.121		&	0.084		& \textbf{	0.055	}	B			\\
128	& \textbf{	0.630	}	&	0.441		&	0.500			& \textbf{	0.614	}	B	& \textbf{	0.957	}	&	0.989			&	0.978		D	& \textbf{	0.994	}	BO	& \textbf{	0.080	}	&	0.136		&	0.119		& \textbf{	0.085	}	B			\\ \hline
\multicolumn{13}{|c|}{ZDT3}																																													\\ \hline
8	&	0.917		& \textbf{	0.971	}	&	0.960			&	Equiv. D			& \textbf{	0.843	}	& \textbf{	0.854	}	B	&	0.868		D	&	Equiv. D			&	0.009		& \textbf{	0.001	}	&	0.004		&	Equiv. D					\\
16	&	0.911		&	0.876	B	& \textbf{	0.963	}		&	Equiv. O			& \textbf{	0.864	}	&	0.899		B	& \textbf{	0.837	}	B	&	Equiv. O			&	0.010		&	0.014		& \textbf{	0.003	}	&	Equiv. O					\\
32	&	0.884		&	0.710		&	0.883		B	& \textbf{	0.931	}		& \textbf{	0.856	}	& \textbf{	0.870	}	B	& \textbf{	0.842	}	B	& \textbf{	0.842	}	BDO	&	0.013		&	0.032		&	0.013	B	& \textbf{	0.008	}				\\
64	&	0.828		&	0.645		&	0.728			& \textbf{	0.854	}		& \textbf{	0.878	}	& \textbf{	0.896	}	B	& \textbf{	0.871	}	BD	& \textbf{	0.873	}	BDO	& \textbf{	0.019	}	&	0.040		&	0.030		& \textbf{	0.016	}	B			\\
128	& \textbf{	0.770	}	&	0.620		&	0.651			& \textbf{	0.773	}	B	& \textbf{	0.887	}	& \textbf{	0.901	}	B	& \textbf{	0.890	}	BD	& \textbf{	0.885	}	BDO	& \textbf{	0.026	}	&	0.043		&	0.039		& \textbf{	0.025	}	B			\\ \hline
\multicolumn{13}{|c|}{ZDT6}																																													\\ \hline
8	&	0.271		& \textbf{	0.398	}	&	0.323			&	Equiv. D			& \textbf{	0.982	}	& \textbf{	0.982	}	B	&	0.994			&	Equiv. D			&	0.171		& \textbf{	0.115	}	&	0.149		&	Equiv. D					\\
16	&	0.275		&	0.295	B	& \textbf{	0.354	}		&	Equiv. O			& \textbf{	0.981	}	& \textbf{	0.970	}	B	&	1.006			&	Equiv. O			&	0.170		&	0.161	B	& \textbf{	0.136	}	&	Equiv. O					\\
32	&	0.239		&	0.123		&	0.240			& \textbf{	0.254	}		& \textbf{	0.989	}	& \textbf{	0.991	}	B	& \textbf{	0.982	}	BD	& \textbf{	0.999	}	BD	&	0.186		&	0.235		&	0.185	B	& \textbf{	0.179	}				\\
64	& \textbf{	0.184	}	&	0.068		&	0.125			& \textbf{	0.178	}	B	& \textbf{	0.985	}	& \textbf{	0.982	}	B	& \textbf{	0.992	}	B	& \textbf{	0.995	}	BO	& \textbf{	0.209	}	&	0.259		&	0.235		&	0.212					\\
128	& \textbf{	0.128	}	&	0.051		&	0.071			& \textbf{	0.124	}	B	& \textbf{	0.991	}	& \textbf{	0.992	}	B	& \textbf{	0.988	}	BD	&	1.003			& \textbf{	0.233	}	&	0.266		&	0.257		& \textbf{	0.235	}	B			\\ \hline

\end{tabular}
}
\caption{Average quality metrics obtained after 30 runs per configuration, for the 4 methods compared: baseline (B), disjoint (D), overlapped (O) and adaptive overlapped (A), using a chromosome length of 2048 dimensions. Acronyms next to values indicate that there is not significant difference with respect to that method for that value. Best values are marked in bold.}
\label{tab:results2048}
\end{table*}






\begin{table*}
\centering
\resizebox{12cm}{!}{
\begin{tabular}{|c||c|c|c|c||c|c|c|c||}
\hline
	&	\multicolumn{4}{|c|}{Average solutions per island}													&	\multicolumn{4}{|c|}{Generations}															\\ \hline

\#Island	&	B		&	D		&	O			&	A			&	B		&	D			&	O			&	A				\\ \hline
\multicolumn{9}{|c|}{ZDT1}																				\\ \hline
8	&	182.700		&	128.500		&	158.167			&	Equiv. D			&	397.367		&	776.967			&	603.667			&	Equiv. D				\\
16	&	132.567		&	43.633		&	76.167			&	Equiv. O			&	567.933		&	908.467			&	838.000			&	Equiv. O				\\
32	&	102.200		&	37.533		&	34.133	D		&	52.033			&	731.067		&	959.067			&	938.167			&	917.000				\\
64	&	81.100		&	46.267		&	26.300			&	30.167		O	&	846.733		&	978.733			&	972.567			&	927.133	B			\\
128	&	77.167		&	57.767		&	35.767			&	29.167		O	&	923.133		&	984.767			&	984.233	D		&	967.933				\\ \hline
\multicolumn{9}{|c|}{ZDT2}																															\\ \hline
8	&	122.700		&	91.633		&	108.833	BD		&	Equiv. D			&	398.167		&	772.000			&	605.100			&	Equiv. D				\\
16	&	94.867		&	28.033		&	62.467			&	Equiv. O			&	574.567		&	908.433			&	840.900			&	Equiv. O				\\
32	&	56.633		&	11.433		&	19.467	D		&	31.567			&	735.867		&	960.167			&	939.200			&	904.900				\\
64	&	33.467		&	10.267		&	9.367	D		&	19.067			&	849.900		&	978.600			&	972.167			&	929.800	B			\\
128	&	22.833		&	12.633		&	7.800			&	13.967		D	&	929.433		&	984.567			&	984.233	D		&	971.700				\\ \hline
\multicolumn{9}{|c|}{ZDT3}																															\\ \hline
8	&	212.900		&	129.233		&	171.367			&	Equiv. D			&	398.633		&	776.000			&	603.333			&	Equiv. D				\\
16	&	176.133		&	43.567		&	90.100			&	Equiv. O			&	573.033		&	908.667			&	838.000			&	Equiv. O				\\
32	&	121.500		&	42.800		&	33.200			&	56.900			&	733.233		&	959.000			&	937.967			&	918.833				\\
64	&	96.500		&	54.267		&	34.900			&	37.800		O	&	849.200		&	977.167			&	972.200			&	933.167				\\
128	&	76.700		&	62.200		&	41.767			&	33.867		D	&	924.000		&	984.733			&	984.367	D		&	966.433				\\ \hline
\multicolumn{9}{|c|}{ZDT6}																															\\ \hline
8	&	42.233		&	28.633		&	35.967	BD		&	Equiv. D			&	396.900		&	773.400			&	603.533			&	Equiv. D				\\
16	&	43.733		&	7.733		&	18.967			&	Equiv. O			&	568.200		&	908.900			&	836.033			&	Equiv. O				\\
32	&	35.533		&	8.033		&	8.633	D		&	16.233		DO	&	735.700		&	958.900			&	938.600			&	926.167				\\
64	&	22.600		&	10.700		&	7.500	D		&	7.533		DO	&	850.267		&	978.067			&	972.467			&	940.267				\\
128	&	18.133		&	15.000	B	&	9.067			&	11.367		D	&	927.233		&	984.667			&	983.933	D		&	969.433				\\ \hline
\end{tabular}
}
\caption{Average number of generations and average number of solutions per island, obtained after 30 runs per configuration, for the 4 methods compared: baseline (B), disjoint (D), overlapped (O) and adaptive overlapped (A), using a chromosome length of 512 dimensions. Acronyms next to values indicate that there is not significant difference with respect to that method for that value.}
\label{tab:sols512}
\end{table*}





\begin{table*}
\centering
\resizebox{12cm}{!}{
\begin{tabular}{|c||c|c|c|c||c|c|c|c||}
\hline
	&	\multicolumn{4}{|c|}{Average solutions per island}													&	\multicolumn{4}{|c|}{Generations}															\\ \hline
%\multicolumn{13}{|c|}{2048 dimensions}																															\\ \hline
\#Island	&	B		&	D		&	O			&	A			&	B		&	D			&	O			&	A				\\ \hline
\multicolumn{9}{|c|}{ZDT1}																					\\ \hline
8	&	137.733		&	47.167		&	119.933	B		&	Equiv. D			&	175.067		&	225.667			&	207.933			&	Equiv. D				\\
16	&	92.633		&	38.233		&	47.533	D		&	Equiv. O			&	204.867		&	238.533			&	232.900			&	Equiv. O				\\
32	&	56.167		&	41.167		&	28.667			&	32.567	O		&	225.433		&	243.833			&	242.533			&	242.000	O			\\
64	&	50.900		&	49.667	B	&	35.600			&	25.367			&	236.500		&	245.967			&	245.700			&	244.033	D			\\
128	&	45.767		&	64.933		&	44.167	B		&	31.867			&	242.800		&	247.000			&	247.000	D		&	245.733				\\ \hline
\multicolumn{9}{|c|}{ZDT2}																															\\ \hline
8	&	64.267		&	22.733		&	56.867	B		&	Equiv. D			&	175.633		&	226.000			&	208.100			&	Equiv. D				\\
16	&	52.300		&	10.733		&	20.067	D		&	Equiv. O			&	205.033		&	238.867			&	233.100			&	Equiv. O				\\
32	&	34.400		&	10.367		&	10.433	D		&	17.900	O		&	225.600		&	244.267			&	242.633			&	242.000	O			\\
64	&	24.600		&	10.900		&	9.267	D		&	12.867	DO		&	236.700		&	246.000			&	245.767			&	244.033	D			\\
128	&	18.667		&	17.833	B	&	10.367			&	12.933	O		&	243.367		&	247.000			&	247.000	d		&	245.833				\\ \hline
\multicolumn{9}{|c|}{ZDT3}																															\\ \hline
8	&	163.767		&	83.367		&	119.467			&	Equiv. D			&	176.533		&	226.400			&	207.700			&	Equiv. D				\\
16	&	108.467		&	49.433		&	47.700	D		&	Equiv. O			&	205.333		&	238.467			&	232.933			&	Equiv. O				\\
32	&	72.133		&	44.333		&	31.833			&	36.533	O		&	225.100		&	243.900			&	242.433			&	242.000	O			\\
64	&	54.867		&	57.600	B	&	40.633			&	29.300			&	236.333		&	245.933			&	245.733			&	244.000	D			\\
128	&	50.600		&	72.133		&	49.033	B		&	34.933			&	243.200		&	247.000			&	247.000	D		&	245.833				\\ \hline
\multicolumn{9}{|c|}{ZDT6}																															\\ \hline
8	&	22.133		&	13.833		&	14.433	D		&	Equiv. D			&	175.733		&	226.100			&	207.767			&	Equiv. D				\\
16	&	20.767		&	15.867		&	13.300	D		&	Equiv. O			&	205.033		&	238.433			&	232.933			&	Equiv. O				\\
32	&	22.600		&	10.700		&	10.033	D		&	11.967	DO		&	225.833		&	243.667			&	242.500			&	242.000	O			\\
64	&	19.033		&	11.267		&	10.533	D		&	10.567	DO		&	236.433		&	245.900			&	245.833			&	244.000	D			\\
128	&	15.800		&	24.500		&	11.600			&	12.233	DO		&	243.800		&	247.000			&	247.000	D		&	245.533				\\ \hline
\end{tabular}
}
\caption{Average number of generations and average number of solutions per island, obtained after 30 runs per configuration, for the 4 methods compared: baseline (B), disjoint (D), overlapped (O) and adaptive overlapped (A), using a chromosome length of 2048 dimensions. Acronyms next to values indicate that there is not significant difference with respect to that method for that value.}
\label{tab:sols2048}
\end{table*}

%DESCRIPTION OF THE RESULTS
Results show that there is a difference in performance when using 512 or 2048 dimensions. Although in some cases the overlapped methods obtain quality metrics values better or significantly equal than the obtained by the baseline, it is clear that 512 dimensions is not a value high enough to apply these methods. This contradicts the results obtained in our previous work in a shared-memory mono-processor version \citep{Garcia16hpmoon}, where the overlapping method obtained better results than the baseline in both dimension lengths. This can be explained because the added migration latency between nodes of the cluster requires more time than the time saved during the crossover and migration. This was not a problem in the mono-processor version, where the individuals shared the same memory space.

However, with 2048 dimensions, results show that dividing the chromosome produces an improvement in all the quality indicators using the adaptive version (Table \ref{tab:results2048}), even improving the overlapping and disjoint methods. Therefore, there exist some kind of limit point in chromosome length where one method will be preferable to another, besides depending on the number of islands and population size.


%EXPLANATION OF THE RESULTS
This can be explained comparing the number of non-dominated solutions
in each island and the average number of generations (Table
\ref{tab:sols512} and Table \ref{tab:sols2048}). With both chromosome
lengths, increasing the number of islands implies more generations
with all the methods (logically, as there are less individuals in each
island). But also, the SLA methods
approach to the number of generations with the baseline when
increasing the number of islands. However, with $L=2048$ the migration
time is enough big to almost not improve the number of generations
with respect to the baseline, even in the lower number of islands, as
it happens with $L=512$. Therefore, more generations do not
necessarily improve the solution of the global PF, but focusing on
different elements of the chromosome. As indicated previously, every
island is unaware of the other islands PFs, and they are trying to
optimize their solutions independently. With respect to the average
number of solutions in each island, there is a clear difference in
number with the baseline with $L=512$, where this value is in the most
of the cases, less than a half. The number of non-dominated solutions
also improves a better Spread indicator, where the baseline almost
always obtains better values. %TENGO QUE COMENTAR MAS LOS RESULTADOS Y
                              %ENTENDERLO TODO Y JUSTIFICARLO
                              %FINAMENTE
%Almost always no es muy preciso - JJ

Finally, as was already observed in \citep{Garcia16hpmoon}, results
show that all the quality indicators decrease in value when the number of islands is increased, as the  sizes of the subpopulations are smaller. This is consistent with the claim by Dorronsoro et al. \citep{Dorronsoro13superlinear}, who found that cooperative co-evolutionary MOEAs work better on bigger populations (more than 100 individuals).


%%%%REAL PROBLEM
%\subsection{Real problem benchmark: Unsupervised Feature Selection}
%\textcolor{red}{A real problem has also been used to BLABLABLA. In this case, the problem used to solve is the unsupervised feature selection. A dataset formed by EEG signals from Brain Computer Interface (BCI) is used as a benchmark \cite{Kimovski15Parallel}, due to the required high computational cost and chromosome length. Individuals in this case are binary vectors indicating if a specific feature is selected or not. To evaluate a set of features different clustering validation indexes (CVIs) can be used.}

%\textcolor{red}{In this case, the two objectives to compare the three methods are the inter-cluster separation ($f1$) and the intra-cluster separation ($f2$). To obtain this measures, and after applying the projection to the dataset, a method based in the one presented  \cite{Kimovski15Parallel} is applied. Starting with a random value of the projection, the distance to the closer one is stored, and so on. After ordering these distances from lower to higher, we consider the intra-cluster distance ($f1$) as the one in the position 1/4 of the list, while the inter-cluster ($f2$) distance is in the position 3/4. The intra-cluster distance objective needs to be minimized, while the inter-cluster is maximized. Figure \ref{fig:cvis} explains this method using a extremely simplified example. Note that the goal of this paper is not obtain better CVI values than other methods, but to study the different overlapping schemes in parallel MOEAs.}

%\begin{figure}
%\centering
%\includegraphics[width=10cm]{clusterCVI.pdf}
%\caption{\textcolor{red}{Method to calculate the two objectives for the Feature Selection problem. Starting from a random $d$ point of the projection of the features, the distance to the closer point is calculated iteratively. After ordering these distances, the one in the position 1/4 (a) is considered the intra-cluster distance, while the one in position 3/4 (b) is considered the inter-cluster one.}}
%\label{fig:cvis}
%\end{figure}


%\textcolor{red}{The Dataset used is ... Due to the high computational cost required for the fitness function, the running time for each run has been set to 4 hours, with two generations before sending the individual to another random island. Table \ref{tab:parametersBCI} summarizes the specific parameters for this experiment (the rest of the parameters are the same than the previous experiment, shown in Table \ref{tab:parameters} ). As there is no optimal PF known for this problem, we have used the paretos obtained from all runs to obtain the optimal PF to apply the IGD quality metric.}

%\textcolor{red}{
%\begin{table*}
%\begin{center}
%\begin{tabular}{|c|c|}
%\hline
%{\em Parameter Name} & {\em Value} \\ \hline
%Number of islands ($P$) & 8, 32 and 128 \\ \hline
%Chromosome size ($L$) & 3600 \\ \hline
%Crossover type & One-point \\ \hline
%Mutation  type & flip\\ \hline
%Generations between migration & 2 \\ \hline
%Runs per configuration & 10 \\ \hline
%\end{tabular}
%\caption{\textcolor{red}{Parameters and operators used in the experiments for Feature Selection problem (the rest of the parameters are the ones described in Table \ref{tab:parameters}).}}
%\label{tab:parametersBCI}
%\end{center}
%\end{table*}
%}






\section{Conclusions}

Problems that require high-performance % problems are not high-performance. They
                          % _require_ high performance Fergu: done
 and deal with a large number of decision variables can leverage the
 division of the decision space that parallel and distributed
 algorithms imply. This can be done in dMOEAs by selective operator application (SLA), that is, dividing the
 chromosome into different parts, each one modified by a different
 island. This paper compares a baseline distributed NSGA-II with three
 different strategies to separate the chromosome (disjoint or
 overlapping parts), using a high number of islands. Results show that
 these methods can achieve better quality metrics than the baseline in
 the same amount of time, with larger chromosome lengths. 

Results also show that when increasing the number of islands, the
adaptive overlapping method significantly improves the results with
respect to the disjoint and overlapped methods. However, this is only
apparent with larger chromosome sizes. Therefore, the length of the
chromosome is also a key factor to take into account in adaptive
methods, and not only the number of islands. Studying this factor with
more types of problems, and new configurations of population size and
chromosome lengths will be addressed in the future.% 
% More discussion should be added. Why is adaptive better? Is the high
% performance really needed? - JJ
% STILL more discussion should be needed. For instance, the last
% paragraph of results begs an explanation of the influence of diversity on
% results - JJ

Also, more distributed implementations in several systems (such as a
GPUs or heterogeneous clusters) with different amounts of
islands/processors, will be used to perform a scalability study of the
different methods, being the transmission time between islands a
relevant issue to address. Other MOEA algorithms available in the literature, such as SPEA or MOEA/D will be compared. Finally, new benchmarks and real problems
will also be  used to validate this approach.  



\section*{Acknowledgements}
This work has been partially funded by projects TIN2014-56494-C4-3-P y TIN2015-67020-P (Spanish Ministry of Economy and Competitivity), PROY-PP2015-06 (Plan Propio 2015 UGR),  and  MSTR (PRY142/14, Fundaci\'on P\'ublica Andaluza Centro de Estudios Andaluces en la IX Convocatoria de Proyectos de Investigaci\'on).




\bibliographystyle{elsarticle-num}
\bibliography{hpmoon-coor}



\end{document}
