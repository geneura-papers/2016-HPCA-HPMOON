% sage_latex_guidelines.tex V1.01, 11 June 2015

\documentclass[Afour,sageh,times]{sagej}


\usepackage{moreverb,url}

\usepackage[colorlinks,bookmarksopen,bookmarksnumbered,citecolor=red,urlcolor=red]{hyperref}

\newcommand\BibTeX{{\rmfamily B\kern-.05em \textsc{i\kern-.025em b}\kern-.08em
T\kern-.1667em\lower.7ex\hbox{E}\kern-.125emX}}

\def\volumeyear{2015}

\begin{document}

\runninghead{Garc\'ia-S\'anchez et al.}

\title{Parallel Multi-objective optimization using co-evolutionary islands and search space overlapping}

\author{Pablo Garc\'ia-S\'anchez, Julio Ortega, Jes\'us Gonz\'alez, Pedro Angel Castillo and Juan Juli\'an Merelo\affilnum{1}}

\affiliation{\affilnum{1}Dept. of Computer Architecture and Computer Technology, University of Granada, Spain}

\corrauth{Pablo Garc\'ia-S\'anchez,
ETS. Ingenierías en Inform\'atica y Telecomunicaci\'on and CITIC-UGR,
C/Periodista Daniel Saucedo s/n,
Universidad de Granada,
18071, Granada, Spain.}

\email{pablogarcia@ugr.es}

\begin{abstract}
This paper...
\end{abstract}

\keywords{Multi-objective algorithms, NSGA-II, Island model, distributed evolutionary algorithms}

\maketitle

\section{Introduction}

Different approaches have been used to parallelize Evolutionary Algorithms (EAs), as each individual can be considered as an independient unit \citep{Alba13parallel}. Classic methods, such as the global parallel EAs (Master-Slave), or the spatially structured algorithms (Island model or Cellular EAs) have been applied successfully in the past \citep{Folino03cellular,Alba02Parallelism}.  In the case of Multi-Objective Evolutionary Algorithms (MOEAs), these approaches \citep{Luna15Survey} need to deal with a whole solution set called Pareto Front (PFs). This imply to use different distribution and sharing mechanisms, as there exist a tension between the speedup achievable from parallelization and the need to globally recombine results to accurately identify the PF \citep{Branke04Parallelizingcone}.


The application of MOEAs to solve real-world problems have gained attention in the last years \citep{Luna15Survey,Mukhopadhyay14Survey}. These kind of problems require a high number of variables, meaning that the EAs need to deal with large individuals and spent significant extra time for crossover, mutation and migration. Different authors have proposed methods to divide the decision space (the chromosome) to improve the performance and solution quality. In this aspect, the co-evolution model is a dimension-distributed model where a high-dimensional problem is divided into lower dimensional problems \citep{Gong15models}, and evolved separately. One example of application of this technique was described in \citep{Kimovski15Parallel}. The method presented in that works involves different workers that evolve sub-populations created and recombined by a master process, which performs different recombination alternatives from the parts returned by worker processes. A high dimensional problem was used to compare these
alternatives. In \citep{Dorronsoro13superlinear}, a distributed coevolutionary island model was used.  Although in both previous works significant speedups were attained, only a low number of nodes were used (8). However, when increasing the number of islands, the division of each section of the chromosome becomes smaller, and the scalability may be affected by obtaining lower quality solutions in the same amount of time.

In \citep{Garcia16hpmoon} we tried to demonstrate that using overlapped sections of the chromosome in a coevolutionary multi-objective island-based algorithm can improve the quality of the solutions in the same computing time when the number of islands increases. We discovered that the most adequate overlapping size chosen varies depending on a relation between the number of islands and the indidivual size. Moreover, the experiments were conducted in a mono-processor and synchronous island model. This motivates us to continue this research line, but this time, proposing a new method to automatically adapt the size of the overlapped sections, but also, performing all the experiments in a real cluster with up to 128 nodes, with extra number of islands.


The hypothesis of this paper is that using overlapped sections of the chromosome, calculated depending the number of islands to use, in a coevolutionary multi-objective island-based algorithm can improve the quality of the solutions in the same computing time with respect to disjoint and partly overlaped methods. To demonstrate this, a new overlapping islands scheme has been compared with a baseline version of a distributed NSGA-II \citep{Deb00NSGAII}, a coevolutionary disjoint section approach, and a fixed-size overlapped method. Different benchmarks problems and different large number of islands have been used. Results show that this new technique can improve different quality indicators in the same amount of time when the problem size is large. 

The rest of the paper is organized as follows: after a background in parallelization in MOEAs, 
the compared algorithms and the used methodology (Section \ref{sec:met}) are presented. 
Then, the results of the experiments are shown (Section \ref{sec:res}), followed by conclusions and suggestions for future work lines.


%%%%%%%%%%%%%%%%%%%%%%%%%%%%%%  STATE OF THE ART  %%%%%%%%%%%%%%%%%%%%%%%%%%%%%
%
\section{State of the Art}
\label{sec:soa}

Since the early 2000s, the distributed and parallel Evolutionary Algorithms have been used, mostly to take the most from systems such as clusters or grids \citep{Talbi08Parallel}. But in the case of MOEAs, the distribution and parallelization is not as direct as in single-objective EAs. This is because in different steps of the algorithm a whole set of dependent solutions, the Pareto Front, should be managed as a whole, spending time in gather all individuals from the different process or islands.

To solve this issue, some authors have proposed the usage of Master-Slave approaches. For example, Durillo et al. \citep{Durillo08masterslave} compared different master-slave approaches TODO DESCRIBIR MEJOR ESTE TRABAJO. The method proposed by Hiroyasu \citep{Hiroyasu07discussion} generates offspring depending on the computation power. TODO Y ESTE

The work by Deb in 2003 was one of the first approaches for distributed MOEAs (dMOEAs). In that work, the dominance of solutions was divided in the islands using a transformation of coordinates. Their authors concluded that dividing the search space is a good idea, but achieve this is not trivial. Other approaches for dividing the search space include objective separation in different processors \citep{Xiao03specialized}, or divide in two populations: one for elite and other search \citep{Wang09parallel}. Martens, in \citep{Martens13asynchronous} generated a Barabasi-Albert network as the island topology, and selection based on migrating individuals from a not-crowded area and acceptance based on diversity. 


More similar approaches to the one presented here were studied in the
next two works, also using cooperative coevolution for high-dimensional problems. Dorronsoro et al. in
\citep{Dorronsoro13superlinear} obtained super-linear performance
in several instances using cooperative coevolution, focusing each island on a portion of the chromosome. However, in some instances the parallel version 
improved the sequential version.  Kimovski et al. \citep{Kimovski15Parallel} proposed a
master-slave method that splits the population into several processors,
each one running in parallel a MOEA that only affects some portion of
the individuals. After a certain number of generations, the master
process receives all the sub-populations to be combined. Different
combination alternatives in this step are compared. Up to 8 processors
were used in the experimental setup. Our approach in this paper does not
broadcast all solutions to all islands for recombination, as previous
works, but only one solution to a random island, needing less
communication time. Besides, the maximum number of islands of
Dorronsoro or Kimovsky approaches was 8, while in this paper we have
used up to 128 islands. 

Lately, some researchers \citep{cheng2015adaptive} have focused on
working on solution space, dividing the Pareto front in different
clusters in order to model it. This {\em divide and conquer} approach
is readily paralellizable and, in the sense of giving the
responsibility of different parts of the search space to different
modules, is similar to the one presented in this paper.

In our previous work \citep{Garcia16hpmoon} we used some of the previous ideas to compare two co-evolutionary dMOEAs. The first one divided the chromosome in $P$ sections, being $P$ the number of islands. Every island $p$ only performed the mutation and crossover in that part ($p_{th}$) of the chromosome, while the fitness was calculated using the whole individual. After certain number of generations, individuals were migrated randomly to other islands. The performance metrics were calculated at the end of the run. The second method, coevolution with overlapping islands, used the $p_{th-1}$ and $p_{th+1}$ sections of each chromosome, besides the $p_{th}$ part. During the same amount of time both methods obtained better results than a baseline algorithm that dealt with the whole chromosome in each island for crossover and mutation. We discovered that the performance using one or another method depends on the number of sections of the individuals and the number of islands used. This motivates us to find a new adaptive method to select this number of sections of the chromosome to use, depending on the number of islands. Moreover, previous experiments were performed in a single processor synchronous island model with a limited number of islands (8, 32 and 128). In this paper, we have used 8, 16, 32, 64 and 128 islands, and this time, using a real cluster. Therefore, at the same time we are proposing an adaptive method to divide the individual search space, we are validating the previous methods.


We will describe how we have tested our approach in the next Section.




%
%%%%%%%%%%%%%%%%%%%%%%%%%%%%%%  METHODOLOGY  %%%%%%%%%%%%%%%%%%%%%%%%%%%%%
%
\section{Methodology and experimental setup}
\label{sec:met}

This section describes the quality indicators used and the methods compared. The chosen quality indicators, are:

\begin{itemize}
\item Hypervolume (HV): measures the area formed by all non-dominated solutions found with, respect to a reference point. Higher values imply better quality of the PF.
\item Inverted Generational distance (IGD): calculates the distance of the obtained set of solutions to the optimal PF. Therefore, this metric requires the optimal PF found in the literature, or the theoretical one. In this metric, the lower the better. %Pablo: theoretical?
\item Spread (S): Measures the spread between solutions, taking into account the euclidean distance between consecutive solutions. As in previous metric, the lower value, the better, as it implies solutions distributed along all the PF.
\end{itemize}

These metrics have been used extensively, especially in some of the papers presented previously in Section \ref{sec:soa}.

%% --------------------------------------------------------------

\subsection{Baseline algorithm}

This algorithm is a regular NSGA-II algorithm distributed along a number $P$ of islands. After a fixed number of generations, one individual is migrated to another random island. At the end of the run, all PFs of all islands are aggregated in a new one to compute the quality measures. This method as been chosen as it do not require synchronization of the global PF every certain number of generations, as other methods described in Section \ref{sec:soa}, and because it can TERMINAAAAAA
%% some justification on why this was chosen - JJ PABLO: Added 
%
%% And some liaison to next subsection - JJ
This method will be compared to different coevolutionary approaches, that are similar to the baseline, with the exception on the decision space to explore in each island.
\subsection{Coevolutionary algorithms compared}
Three different coevolutionary methods have also been used. In all these methods, each island only performs crossover and mutation in specific sections of the individuals.

As in the baseline, every certain number of generations, an individual is migrated to another random island.  In the new island, this individual will be considered as one of the others in the island, crossed and mutated in the same way, depending of the island identifier. Note that, on the contrary of other works such as the ones described by \cite{Talbi08Parallel} all the islands deal with complete chromosomes for fitness calculation, so our approach can deal with no decomposable problems. 

\subsubsection{Coevolution with disjoint islands} 
In this approach, each individual of size $L$ is split into $P$ chunks of size $L/P$. Every island $p$ only performs crossover and mutation on the $p_{th}$ part of the individuals. Figure \ref{fig:disjoint} describes this approach.  

\begin{figure*}
\includegraphics[width=12cm]{islandDisjoint.jpg}
\caption{Disjoint algorithm: every island $p$ only modifies the $p_{th}$ components (in grey) of the individuals.}
\label{fig:disjoint}
\end{figure*}

\subsubsection{Coevolution with overlapping islands}
This approach is similar to the previous one, but every island also uses the $p+1$ and $p-1$ (module size) chunks of the individual for crossover and migration. Therefore, some kind of overlapping of the crossed and mutated parts exists between islands. Figure \ref{fig:overlapping} shows the affected parts of the individuals in
each island. 

\begin{figure*}
\includegraphics[width=12cm]{islandNoDisjoint.jpg}
\caption{Overlapping algorithm: every island $p$ modifies the  $p+1$,
  $p_{th}$ and $p-1$  components (in grey) of the individuals.}
  \label{fig:overlapping}
\end{figure*}

\subsubsection{Adaptive coevolution with overlapping islands} 
As in the previous method, the sections to address are overlapped, but instead using one extra section on each side of the $p$ section ($p+1$ and $p-1$), it uses $c$ fragments on each side ($p+c$ y $p-c$), being $c$ a value depending on the number of islands. The results shown in \citep{Garcia16evostar} have been used as a base to calculate $c$. In that work, the overlapped method ($c=1$) obtained better results when the number of islands were >8. Therefore, we have used the formula $c=round(0.2*P-1)/2$ to calculate the extra sections to overlap. %TODO TERMINA Y PONER MAS EJEMPLOS Y DECIR QUE ESTA ES UNA PRIMERA APROXIMACION

% --------------------------------------------------------------


%%%%%%%%%%%%%%%%%%%%%%%%%%  EXPERIMENTS AND RESULTS  %%%%%%%%%%%%%%%%%%%%%%%%%%
%
\section{Experiments and Results}
\label{sec:res}

The four approaches have been run with two different chromosome lengths ($L$): 512 and 2048. Different number of islands ($P$) have also been compared: 8, 32 and 128. This maximum number of islands have also been  used in previous work in the literature \citep{Martens13asynchronous}. The crossover and mutation chosen, SBX and polynomial, have been used previously by other authors in \citep{Durillo08masterslave}. 

ZDT \citep{zdt2000a} has been chosen as a benchmark, since it is the most widely used in this area \citep{Deb03distributed,Martens13asynchronous,Wang09parallel,Durillo08masterslave}. The optimal PF distribution used for comparison has 1000 solutions \footnote{Optimal PFs are available at:   \url{http://www.tik.ee.ethz.ch/sop/download/supplementary/testproblems/}.}. 





The termination criterion used is the execution time: 25 seconds for dimension 512 and 100 (four times more) for 2048. We have used the time instead the number of evaluations firstly because our hypothesis argue that the time saved in crossover in mutation can be spent on improving the sub-populations and more operations and migrations can be achieved. Also, we are using different number of islands (with different sub-population sizes) that could lead to different execution times, so it would be difficult to compare different times and quality solutions at the same time.

\begin{table*}
\begin{center}
\begin{tabular}{|c|c|}
\hline
{\em Parameter Name} & {\em Value} \\ \hline
Number of islands ($P$) & 8, 32 and 128 \\ \hline
Chromosome size ($L$) & 512 and 2048 \\ \hline
Execution time (s) & 25 (for 512) and 100 (for 2048) \\ \hline \hline
Global population size ($N$) & 1024 \\ \hline
Selection type & Binary Tournament Selection \\ \hline
Replacement type & Generational \\ \hline 
Crossover type & SBX \\ \hline
Mutation  type & Polynomial\\ \hline
Mutation probability & 1/$L$ \\ \hline
Generations between migration & 5 \\ \hline
Individuals per migration & 1 \\ \hline
Selection for migration & Binary Tournament\\ \hline
Runs per configuration & 30 \\ \hline
\end{tabular}
\caption{Parameters and operators used in the experiments.}
\label{tab:parameters}
\end{center}
\end{table*}

The ECJ framework \citep{ECJ} has been used to run the
experiments. Specific operators have been developed as new modules for
ECJ, and they can be downloaded from our GitHub repository under a
LGPL V3 License
\footnote{\url{https://https://github.com/hpmoon/hpmoon-islands}}. The
island model has been executed synchronously, using the ECJ Internal
Island Model, in a CentOS 5.4 machine with Intel(R) Xeon(R) CPU E5520
@2.27GHz, 16 GB RAM, with Java Version 1.6.0\_16. As in this paper we
only focus on the behaviour of our model in a single machine scenario
(the islands share the same memory and processor), the migration time
will not be as important factor as in a real distributed system. This
task will be addressed in future works. %Pablo: no sé si decir esto
                                %aquí% 
%
%

Different metrics, explained in the previous section, have been used to calculate the quality of the obtained PFs in each configuration. As some of the metrics require a reference point to be calculated (such as HV), the point (1,9) has been chosen as reference, as none of the generated PFs in all runs are dominated by it. Metrics are then normalized with respect to that point. A Kruskall-Wallis significance test has been performed to the metrics of all runs of the configurations, as the Kolmogorov-Smirnov test detected non-normal distributions. The average results for each configuration are shown in Table \ref{tab:results512} (for 512 dimensions) and \ref{tab:results2048} (for 2048).


\begin{table*}
\centering
\resizebox{14cm}{!}{
\begin{tabular}{|c||c|c|c|c||c|c|c|c||c|c|c|c||}
\hline
	&	\multicolumn{4}{|c|}{HV}													&	\multicolumn{4}{|c|}{Spread}														&	\multicolumn{4}{|c|}{IGD}														\\ \hline
%\multicolumn{13}{|c|}{2048 dimenOionO}																																													\\ \hline
\#Island	&	B		&	D		&	O			&	A			&	B		&	D			&	O			&	A			&	B		&	D		&	O		&	A					\\ \hline
\multicolumn{13}{|c|}{ZDT1}																																													\\ \hline
8	&	0.900		&	0.816		& \textbf{	0.926	}		&	Equiv. D			& \textbf{	0.729	}	&	1.041			&	0.898			&	Equiv. D			&	0.013		&	0.029		& \textbf{	0.007	}	&	Equiv. D					\\
16	& \textbf{	0.880	}	&	0.675		&	0.839			&	Equiv. O			& \textbf{	0.721	}	&	0.897			&	0.983		D	&	Equiv. O			& \textbf{	0.017	}	&	0.062		&	0.025		&	Equiv. O					\\
32	& \textbf{	0.829	}	&	0.618		&	0.698			&	0.790			& \textbf{	0.763	}	&	0.879			&	0.886		D	&	0.913	DO		& \textbf{	0.027	}	&	0.076		&	0.056		&	0.036					\\
64	& \textbf{	0.769	}	&	0.596		&	0.628			&	0.736			& \textbf{	0.820	}	&	0.897			&	0.872			&	0.890	DO		& \textbf{	0.040	}	&	0.080		&	0.071		&	0.047					\\
128	& \textbf{	0.707	}	&	0.585		&	0.598			&	0.668			& \textbf{	0.850	}	&	0.956			&	0.883			&	0.910	O		& \textbf{	0.053	}	&	0.083		&	0.079		&	0.062					\\ \hline
\multicolumn{13}{|c|}{ZDT2}																																													\\ \hline
8	&	0.830		&	0.786		& \textbf{	0.851	}		&	Equiv. D			& \textbf{	0.866	}	&	0.996			&	0.987		D	&	Equiv. D			&	0.023		&	0.037		& \textbf{	0.017	}	&	Equiv. D					\\
16	& \textbf{	0.810	}	&	0.601		&	0.776			&	Equiv. O			& \textbf{	0.916	}	&	0.990			&	1.005		D	&	Equiv. O			& \textbf{	0.029	}	&	0.091		&	0.040		&	Equiv. O					\\
32	& \textbf{	0.756	}	&	0.498		&	0.617			&	0.682			& \textbf{	0.976	}	& \textbf{	0.976	}	B	& \textbf{	0.978	}	BD	&	1.054			& \textbf{	0.045	}	&	0.119		&	0.086		&	0.068					\\
64	& \textbf{	0.668	}	&	0.451		&	0.504			&	0.601			&	1.002		& \textbf{	0.976	}		& \textbf{	0.974	}	D	&	1.025	B		& \textbf{	0.070	}	&	0.132		&	0.117		&	0.091					\\
128	& \textbf{	0.576	}	&	0.434		&	0.452			&	0.526			&	1.002		&	1.023			& \textbf{	0.978	}		&	1.015	BD		& \textbf{	0.096	}	&	0.137		&	0.132		&	0.110					\\ \hline
\multicolumn{13}{|c|}{ZDT3}																																													\\ \hline
8	&	0.928		&	0.838		& \textbf{	0.951	}		&	Equiv. D			& \textbf{	0.869	}	&	1.028			&	0.983		D	&	Equiv. D			& \textbf{	0.008	}	&	0.018		&	0.005		&	Equiv. D					\\
16	& \textbf{	0.903	}	&	0.715		&	0.870			&	Equiv. O			& \textbf{	0.846	}	&	0.899			&	0.960			&	Equiv. O			& \textbf{	0.011	}	&	0.032		&	0.014		&	Equiv. O					\\
32	& \textbf{	0.857	}	&	0.655		&	0.737			&	0.824			& \textbf{	0.845	}	&	0.879			&	0.873		D	&	0.881	DO		& \textbf{	0.016	}	&	0.039		&	0.030		&	0.019					\\
64	& \textbf{	0.796	}	&	0.632		&	0.662			&	0.761			& \textbf{	0.859	}	&	0.887			&	0.885		D	&	0.879	BDO		& \textbf{	0.023	}	&	0.042		&	0.038		&	0.027					\\
128	& \textbf{	0.738	}	&	0.620		&	0.633			&	0.705			& \textbf{	0.882	}	&	0.968			& \textbf{	0.888	}	B	&	0.911			& \textbf{	0.029	}	&	0.044		&	0.042		&	0.033					\\ \hline
\multicolumn{13}{|c|}{ZDT6}																																													\\ \hline
8	&	0.268		&	0.223		& \textbf{	0.298	}		&	Equiv. D			& \textbf{	0.988	}	& \textbf{	0.972	}	B	& \textbf{	0.996	}	B	&	Equiv. D			& \textbf{	0.173	}	&	0.191		&	0.160		&	Equiv. D					\\
16	& \textbf{	0.243	}	&	0.113		&	0.219			&	Equiv. O			& \textbf{	0.995	}	& \textbf{	0.987	}	B	& \textbf{	0.987	}	BD	&	Equiv. O			& \textbf{	0.184	}	&	0.240		&	0.193		&	Equiv. O					\\
32	& \textbf{	0.196	}	&	0.071		&	0.115			&	0.162			& \textbf{	0.998	}	& \textbf{	0.989	}	B	& \textbf{	0.988	}	BD	&	1.005	B		& \textbf{	0.204	}	&	0.257		&	0.238		&	0.220					\\
64	& \textbf{	0.145	}	&	0.056		&	0.075			&	0.120			&	0.996		&	0.986		B	& \textbf{	0.983	}	D	&	0.998	B		& \textbf{	0.226	}	&	0.264		&	0.255		&	0.235					\\
128	& \textbf{	0.104	}	&	0.049		&	0.055			&	0.084			&	0.993		&	1.007			& \textbf{	0.979	}		&	0.998	BD		& \textbf{	0.244	}	&	0.266		&	0.263		&	0.251					\\ \hline

\end{tabular}
}
\caption{Quality metrics obtained after 30 runs per configuration, for the 4 methods compared: baseline (B), disjoint (D), overlapped (O) and adaptive overlapped (A), using a chromosome length of 512 dimensions. Acronyms next to values indicate that there is not significant difference with respect to that method for that value. Best values are marked in bold.}
\label{tab:results512}
\end{table*}



\begin{table*}
\centering
\resizebox{16cm}{!}{
\begin{tabular}{|c||c|c|c|c||c|c|c|c||c|c|c|c||}
\hline
	&	\multicolumn{4}{|c|}{HV}													&	\multicolumn{4}{|c|}{Spread}														&	\multicolumn{4}{|c|}{IGD}														\\ \hline

\#Island	&	B		&	D		&	O			&	A			&	B		&	D			&	O			&	A			&	B		&	D		&	O		&	A					\\ \hline
\multicolumn{13}{|c|}{ZDT1}																																													\\ \hline
8	&	0.891		& \textbf{	0.953	}	&	0.937			&	Equiv. D			& \textbf{	0.681	}	& \textbf{	0.635	}	B	&	0.661		D	&	Equiv. D			&	0.015		& \textbf{	0.002	}	&	0.005		&	Equiv. D					\\
16	&	0.884		&	0.850		& \textbf{	0.942	}		&	Equiv. O			& \textbf{	0.705	}	&	0.908			& \textbf{	0.670	}	B	&	Equiv. O			&	0.016		&	0.022		& \textbf{	0.004	}	&	Equiv. O					\\
32	&	0.851		&	0.674		&	0.859		B	& \textbf{	0.900	}		& \textbf{	0.754	}	&	0.868			&	0.826		D	& \textbf{	0.763	}	B	&	0.023		&	0.062		&	0.020	B	& \textbf{	0.012	}				\\
64	& \textbf{	0.800	}	&	0.608		&	0.697			& \textbf{	0.824	}	B	& \textbf{	0.808	}	&	0.880			& \textbf{	0.861	}	B	& \textbf{	0.823	}	B	&	0.033		&	0.078		&	0.056		& \textbf{	0.027	}				\\
128	&	0.735		&	0.582		&	0.613			& \textbf{	0.745	}		& \textbf{	0.841	}	&	0.888			&	0.878		D	&	0.865		O	&	0.047		&	0.084		&	0.075		& \textbf{	0.043	}				\\ \hline
\multicolumn{13}{|c|}{ZDT2}																																													\\ \hline
8	&	0.832		& \textbf{	0.895	}	&	0.869			&	Equiv. D			& \textbf{	0.849	}	& \textbf{	0.886	}	B	&	0.853		D	&	Equiv. D			&	0.023		& \textbf{	0.006	}	&	0.013		&	Equiv. D					\\
16	&	0.831		&	0.833	B	& \textbf{	0.884	}		&	Equiv. O			& \textbf{	0.810	}	&	1.001			& \textbf{	0.802	}	B	&	Equiv. O			&	0.023		&	0.022	B	& \textbf{	0.009	}	&	Equiv. O					\\
32	&	0.800		&	0.628		&	0.800		B	& \textbf{	0.817	}		& \textbf{	0.848	}	&	0.974			&	0.983		D	&	0.908			&	0.031		&	0.082		&	0.032	B	& \textbf{	0.027	}				\\
64	& \textbf{	0.729	}	&	0.491		&	0.623			&	0.716			& \textbf{	0.909	}	&	0.967			&	0.979		D	& \textbf{	0.997	}	BD	& \textbf{	0.052	}	&	0.121		&	0.084		& \textbf{	0.055	}	B			\\
128	& \textbf{	0.630	}	&	0.441		&	0.500			& \textbf{	0.614	}	B	& \textbf{	0.957	}	&	0.989			&	0.978		D	& \textbf{	0.994	}	BO	& \textbf{	0.080	}	&	0.136		&	0.119		& \textbf{	0.085	}	B			\\ \hline
\multicolumn{13}{|c|}{ZDT3}																																													\\ \hline
8	&	0.917		& \textbf{	0.971	}	&	0.960			&	Equiv. D			& \textbf{	0.843	}	& \textbf{	0.854	}	B	&	0.868		D	&	Equiv. D			&	0.009		& \textbf{	0.001	}	&	0.004		&	Equiv. D					\\
16	&	0.911		&	0.876	B	& \textbf{	0.963	}		&	Equiv. O			& \textbf{	0.864	}	&	0.899		B	& \textbf{	0.837	}	B	&	Equiv. O			&	0.010		&	0.014		& \textbf{	0.003	}	&	Equiv. O					\\
32	&	0.884		&	0.710		&	0.883		B	& \textbf{	0.931	}		& \textbf{	0.856	}	& \textbf{	0.870	}	B	& \textbf{	0.842	}	B	& \textbf{	0.842	}	BDO	&	0.013		&	0.032		&	0.013	B	& \textbf{	0.008	}				\\
64	&	0.828		&	0.645		&	0.728			& \textbf{	0.854	}		& \textbf{	0.878	}	& \textbf{	0.896	}	B	& \textbf{	0.871	}	BD	& \textbf{	0.873	}	BDO	& \textbf{	0.019	}	&	0.040		&	0.030		& \textbf{	0.016	}	B			\\
128	& \textbf{	0.770	}	&	0.620		&	0.651			& \textbf{	0.773	}	B	& \textbf{	0.887	}	& \textbf{	0.901	}	B	& \textbf{	0.890	}	BD	& \textbf{	0.885	}	BDO	& \textbf{	0.026	}	&	0.043		&	0.039		& \textbf{	0.025	}	B			\\ \hline
\multicolumn{13}{|c|}{ZDT6}																																													\\ \hline
8	&	0.271		& \textbf{	0.398	}	&	0.323			&	Equiv. D			& \textbf{	0.982	}	& \textbf{	0.982	}	B	&	0.994			&	Equiv. D			&	0.171		& \textbf{	0.115	}	&	0.149		&	Equiv. D					\\
16	&	0.275		&	0.295	B	& \textbf{	0.354	}		&	Equiv. O			& \textbf{	0.981	}	& \textbf{	0.970	}	B	&	1.006			&	Equiv. O			&	0.170		&	0.161	B	& \textbf{	0.136	}	&	Equiv. O					\\
32	&	0.239		&	0.123		&	0.240			& \textbf{	0.254	}		& \textbf{	0.989	}	& \textbf{	0.991	}	B	& \textbf{	0.982	}	BD	& \textbf{	0.999	}	BD	&	0.186		&	0.235		&	0.185	B	& \textbf{	0.179	}				\\
64	& \textbf{	0.184	}	&	0.068		&	0.125			& \textbf{	0.178	}	B	& \textbf{	0.985	}	& \textbf{	0.982	}	B	& \textbf{	0.992	}	B	& \textbf{	0.995	}	BO	& \textbf{	0.209	}	&	0.259		&	0.235		&	0.212					\\
128	& \textbf{	0.128	}	&	0.051		&	0.071			& \textbf{	0.124	}	B	& \textbf{	0.991	}	& \textbf{	0.992	}	B	& \textbf{	0.988	}	BD	&	1.003			& \textbf{	0.233	}	&	0.266		&	0.257		& \textbf{	0.235	}	B			\\ \hline

\end{tabular}
}
\caption{Quality metrics obtained after 30 runs per configuration, for the 4 methods compared: baseline (B), disjoint (D), overlapped (O) and adaptive overlapped (A), using a chromosome length of 2048 dimensions. Acronyms next to values indicate that there is not significant difference with respect to that method for that value. Best values are marked in bold.}
\label{tab:results2048}
\end{table*}






\begin{table*}
\centering
\resizebox{12cm}{!}{
\begin{tabular}{|c||c|c|c|c||c|c|c|c||}
\hline
	&	\multicolumn{4}{|c|}{Average solutions per island}													&	\multicolumn{4}{|c|}{Generations}															\\ \hline

\#Island	&	B		&	D		&	O			&	A			&	B		&	D			&	O			&	A				\\ \hline
\multicolumn{9}{|c|}{ZDT1}																				\\ \hline
8	&	182.700		&	128.500		&	158.167			&	Equiv. D			&	397.367		&	776.967			&	603.667			&	Equiv. D				\\
16	&	132.567		&	43.633		&	76.167			&	Equiv. O			&	567.933		&	908.467			&	838.000			&	Equiv. O				\\
32	&	102.200		&	37.533		&	34.133	D		&	52.033			&	731.067		&	959.067			&	938.167			&	917.000				\\
64	&	81.100		&	46.267		&	26.300			&	30.167		O	&	846.733		&	978.733			&	972.567			&	927.133	B			\\
128	&	77.167		&	57.767		&	35.767			&	29.167		O	&	923.133		&	984.767			&	984.233	D		&	967.933				\\ \hline
\multicolumn{9}{|c|}{ZDT2}																															\\ \hline
8	&	122.700		&	91.633		&	108.833	BD		&	Equiv. D			&	398.167		&	772.000			&	605.100			&	Equiv. D				\\
16	&	94.867		&	28.033		&	62.467			&	Equiv. O			&	574.567		&	908.433			&	840.900			&	Equiv. O				\\
32	&	56.633		&	11.433		&	19.467	D		&	31.567			&	735.867		&	960.167			&	939.200			&	904.900				\\
64	&	33.467		&	10.267		&	9.367	D		&	19.067			&	849.900		&	978.600			&	972.167			&	929.800	B			\\
128	&	22.833		&	12.633		&	7.800			&	13.967		D	&	929.433		&	984.567			&	984.233	D		&	971.700				\\ \hline
\multicolumn{9}{|c|}{ZDT3}																															\\ \hline
8	&	212.900		&	129.233		&	171.367			&	Equiv. D			&	398.633		&	776.000			&	603.333			&	Equiv. D				\\
16	&	176.133		&	43.567		&	90.100			&	Equiv. O			&	573.033		&	908.667			&	838.000			&	Equiv. O				\\
32	&	121.500		&	42.800		&	33.200			&	56.900			&	733.233		&	959.000			&	937.967			&	918.833				\\
64	&	96.500		&	54.267		&	34.900			&	37.800		O	&	849.200		&	977.167			&	972.200			&	933.167				\\
128	&	76.700		&	62.200		&	41.767			&	33.867		D	&	924.000		&	984.733			&	984.367	D		&	966.433				\\ \hline
\multicolumn{9}{|c|}{ZDT6}																															\\ \hline
8	&	42.233		&	28.633		&	35.967	BD		&	Equiv. D			&	396.900		&	773.400			&	603.533			&	Equiv. D				\\
16	&	43.733		&	7.733		&	18.967			&	Equiv. O			&	568.200		&	908.900			&	836.033			&	Equiv. O				\\
32	&	35.533		&	8.033		&	8.633	D		&	16.233		DO	&	735.700		&	958.900			&	938.600			&	926.167				\\
64	&	22.600		&	10.700		&	7.500	D		&	7.533		DO	&	850.267		&	978.067			&	972.467			&	940.267				\\
128	&	18.133		&	15.000	B	&	9.067			&	11.367		D	&	927.233		&	984.667			&	983.933	D		&	969.433				\\ \hline
\end{tabular}
}
\caption{Average number of generations and average number of solutions per island, obtained after 30 runs per configuration, for the 4 methods compared: baseline (B), disjoint (D), overlapped (O) and adaptive overlapped (A), using a chromosome length of 512 dimensions. Acronyms next to values indicate that there is not significant difference with respect to that method for that value.}
\label{tab:sols512}
\end{table*}





\begin{table*}
\centering
\resizebox{12cm}{!}{
\begin{tabular}{|c||c|c|c|c||c|c|c|c||}
\hline
	&	\multicolumn{4}{|c|}{Average solutions per island}													&	\multicolumn{4}{|c|}{Generations}															\\ \hline
%\multicolumn{13}{|c|}{2048 dimensions}																															\\ \hline
\#Island	&	B		&	D		&	O			&	A			&	B		&	D			&	O			&	A				\\ \hline
\multicolumn{9}{|c|}{ZDT1}																					\\ \hline
8	&	137.733		&	47.167		&	119.933	B		&	Equiv. D			&	175.067		&	225.667			&	207.933			&	Equiv. D				\\
16	&	92.633		&	38.233		&	47.533	D		&	Equiv. O			&	204.867		&	238.533			&	232.900			&	Equiv. O				\\
32	&	56.167		&	41.167		&	28.667			&	32.567	O		&	225.433		&	243.833			&	242.533			&	242.000	O			\\
64	&	50.900		&	49.667	B	&	35.600			&	25.367			&	236.500		&	245.967			&	245.700			&	244.033	D			\\
128	&	45.767		&	64.933		&	44.167	B		&	31.867			&	242.800		&	247.000			&	247.000	D		&	245.733				\\ \hline
\multicolumn{9}{|c|}{ZDT2}																															\\ \hline
8	&	64.267		&	22.733		&	56.867	B		&	Equiv. D			&	175.633		&	226.000			&	208.100			&	Equiv. D				\\
16	&	52.300		&	10.733		&	20.067	D		&	Equiv. O			&	205.033		&	238.867			&	233.100			&	Equiv. O				\\
32	&	34.400		&	10.367		&	10.433	D		&	17.900	O		&	225.600		&	244.267			&	242.633			&	242.000	O			\\
64	&	24.600		&	10.900		&	9.267	D		&	12.867	DO		&	236.700		&	246.000			&	245.767			&	244.033	D			\\
128	&	18.667		&	17.833	B	&	10.367			&	12.933	O		&	243.367		&	247.000			&	247.000	d		&	245.833				\\ \hline
\multicolumn{9}{|c|}{ZDT3}																															\\ \hline
8	&	163.767		&	83.367		&	119.467			&	Equiv. D			&	176.533		&	226.400			&	207.700			&	Equiv. D				\\
16	&	108.467		&	49.433		&	47.700	D		&	Equiv. O			&	205.333		&	238.467			&	232.933			&	Equiv. O				\\
32	&	72.133		&	44.333		&	31.833			&	36.533	O		&	225.100		&	243.900			&	242.433			&	242.000	O			\\
64	&	54.867		&	57.600	B	&	40.633			&	29.300			&	236.333		&	245.933			&	245.733			&	244.000	D			\\
128	&	50.600		&	72.133		&	49.033	B		&	34.933			&	243.200		&	247.000			&	247.000	D		&	245.833				\\ \hline
\multicolumn{9}{|c|}{ZDT6}																															\\ \hline
8	&	22.133		&	13.833		&	14.433	D		&	Equiv. D			&	175.733		&	226.100			&	207.767			&	Equiv. D				\\
16	&	20.767		&	15.867		&	13.300	D		&	Equiv. O			&	205.033		&	238.433			&	232.933			&	Equiv. O				\\
32	&	22.600		&	10.700		&	10.033	D		&	11.967	DO		&	225.833		&	243.667			&	242.500			&	242.000	O			\\
64	&	19.033		&	11.267		&	10.533	D		&	10.567	DO		&	236.433		&	245.900			&	245.833			&	244.000	D			\\
128	&	15.800		&	24.500		&	11.600			&	12.233	DO		&	243.800		&	247.000			&	247.000	D		&	245.533				\\ \hline
\end{tabular}
}
\caption{Average number of generations and average number of solutions per island, obtained after 30 runs per configuration, for the 4 methods compared: baseline (B), disjoint (D), overlapped (O) and adaptive overlapped (A), using a chromosome length of 2048 dimensions. Acronyms next to values indicate that there is not significant difference with respect to that method for that value.}
\label{tab:sols2048}
\end{table*}

%DESCRIPTION OF THE RESULTS
Results show that there is a difference in performance when using 512 or 2048 dimensions. Although in some cases the overlapped methods obtain quality metrics values better or significantly equal than the obtained by the baseline, it is clear that 512 dimensions is not a value high enough to apply these methods. This contradicts the results obtained in our previous work in a shared-memory mono-processor version \citep{Garcia16hpmoon}, where the overlapping method obtained best results than the baseline in both dimension lengths. This can be explained because the added migration latency between nodes requires more time than the time saved during the crossover and migration.

However, with 2048 dimensions, results show that dividing the chromosome produces an improvement in all the quality indicators using the adaptive version (Table \ref{tab:results2048}), even improving the overlapping and disjoint methods. Therefore, there exist some kind of limit point in chromosome length where one method will be preferable to another, depending the number of islands, population size and length.


%EXPLANATION OF THE RESULTS
This can be explained comparing the number non-dominated solutions in each island and the average number of generations (Table \ref{tab:sols512} and Table \ref{tab:sols2048}). With both chromosome lengths, increasing the number of islands implies more generations with all the methods (logically, as there are less individuals in each island). But also, the co-evolutionary methods approach to the number of generations with the baseline when increasing the number of islands. However, with $L=2048$ the migration time is enough big to almost not improve the number of generations with respect to the baseline, even in the lower number of islands, as it happens with $L=512$. Therefore, more generations do not necessarily improve the solution of the global PF, but focusing in different elements of the chromosome. As indicated previously, every island is unaware of the other islands PFs, and they are trying to optimize their solutions independently. With respect to the average number of solution in each island, there is a clear difference in number with the baseline with $L=512$, where this value is in the most of the cases, less than a half. The number of non-dominated solutions also improves a better Spread indicator, where the baseline almost always obtains better values %TENGO QUE COMENTAR MAS LOS RESULTADOS Y ENTENDERLO TODO Y JUSTIFICARLO FINAMENTE

Finally, as happened in \citep{Garcia16hpmoon}, results show that all the quality indicators lowers when the number of islands is increased, as the subpopulations are smaller. This is consistent with the claim by Durillo in \citep{Dorronsoro13superlinear}, who found that cooperative coevolutionary MOEAs work better on bigger populations (more than 100 individuals).



%Surprisingly, the average
%number of solutions of PFs per island in the baseline configuration is significantly
%higher in some configurations (mostly with 128 islands), but their
%combination in the final PF never attains higher values than Disjoint
%and Overlapping methods. 


%configurations, with respect to the baseline.  Paying attention to the disjoint and
%overlapping methods, only when using 8 islands the disjoint method
%always attain better metrics, and even more times with the 2048 chromosome
%length. This can be explained because only 1/8 of the individual is
%changed in each island, while in the overlapping, 3/8 of the island
%may be more similar to the baseline configuration. 
%




\section{Conclusions}

%High-performance problems that require a large number of decision
%variables can leverage the division of the decision space that parallel
%and distributed algorithms imply. This can be done in dMOEAs by dividing the
%chromosome into different parts, each one modified by a different
%island. This paper compares a baseline distributed NSGA-II with two
%different strategies to separate the chromosome (disjoint or overlapping
%parts), using a high number of islands. Results show that these 
%methods can achieve better quality metrics
%than the baseline in the same amount of time. This can be explained 
%because of the time reduction in crossover and mutation in the chromosomes,
%producing a higher number of generations and more modifications of the 
%solutions of the PF in each sub-population, and therefore improving
%quality of the global PF.%

%Results also show that when increasing the number of islands, the overlapping 
%method significantly improves the results with respect to the disjoint method.
%Therefore, the relation between the number of islands used, the global population size,
%and the length of the chromosome may be a key factor to decide if using the 
%overlapping method or the disjoint one. Studying this factor with more types 
%of problems, and new configurations of population size and chromosome lengths will be addressed in the future.%
%

%Also, distributed implementations in several systems (such as a GRID or cluster) with more different
%quantities of islands/processors, will be used to perform a scalability study of the different
%methods, being the transmission time between islands a relevant issue to
%address. New benchmarks and real problems will also be  used to validate
%our approach. 



\begin{acks}
Acknowledgements
\end{acks}

TIN2014-56494-C4-3-P y TIN2015-67020-P (Ministerio de Economía y Competitividad), PROY-PP2015-06 (Plan Propio 2015 UGR),  y MSTR (PRY142/14, Fundación Pública Andaluza Centro de Estudios Andaluces en la IX Convocatoria de Proyectos de Investigación).

\bibliographystyle{SageH}

\bibliography{hpmoon-hpca}
%\begin{thebibliography}{99}
%\bibitem[Kopka and Daly(2003)]{R1}
%Kopka~H and Daly~PW (2003) \textit{A Guide to \LaTeX}, 4th~edn.
%Addison-Wesley.%

%\bibitem[Lamport(1994)]{R2}
%Lamport~L (1994) \textit{\LaTeX: a Document Preparation System},
%2nd~edn. Addison-Wesley.%

%\bibitem[Mittelbach and Goossens(2004)]{R3}
%Mittelbach~F and Goossens~M (2004) \textit{The \LaTeX\ Companion},
%2nd~edn. Addison-Wesley.%

%\end{thebibliography}

\end{document}
